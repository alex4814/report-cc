\documentclass[12pt,letter]{article}
\usepackage[utf8]{inputenc}
\usepackage{amsmath}
\usepackage{amsfonts}
\usepackage{amssymb}
\usepackage{graphicx}

\usepackage{hyperref}
\usepackage[numbers,sort&compress]{natbib}
\usepackage{hypernat}

\bibliographystyle{plain}

\author{Ruisheng Fu\\21421190\\CCNT Lab.
\and
Guanyu Guo\\21421189\\LIST Lab.
\and
Chen Sui\\21421183}
\title{NoSQL in Cloud Computing}
\date{\today}

\begin{document}
\maketitle
\tableofcontents

\begin{abstract}
A lot of changes in database management system has been made since the inception of cloud computing. Such critical and increasing needs within cloud computing as scalability, elasticity and  processing a huge amount of data can be fulfilled by the NoSQL databases as opposed to RDBMS\footnote{Relational Database Management System}. In this report we have dived into several primary NoSQL databases used in leading cloud computing vendors, summarizing and discussing detailed techniques as well as analysis with comparisons.
\end{abstract}

\section{Introduction}
Cloud computing has been a evolving computing terminology that is very much in the public eye. It is responsible for managing and grouping remote servers that allow data storage and online access to various services. 


\section{Main Research Streams}

\section{Conclusion}
ddf.\cite{Chang2006}
\section{Help}
\subsection{1}
Linking words are important. If you are grouping together writers with similar opinions, you would use words or phrases such as:

similarly, in addition, also, again

More importantly, if there is disagreement, you need to indicate clearly that you are aware of this by the use of linkers such as:

however, on the other hand, conversely, nevertheless

At the end of the review you should include a summary of what the literature implies, which again links to your hypothesis or main question.
\subsection{2}
What reporting verbs should I use?
Avoid over-using "states" and "says". You may need to use tentative or evaluative verbs.
• Tentative verbs are often used to show that findings are incomplete or difficult to generalise from.
e.g. Research suggests that a majority of people prefer email to... (Mahlab 1995). Wang (2003) indicates that such results are not necessarily...
• Evaluative verbs can pack in extra meaning by incorporating your evaluation of the text. e.g. Jacob concedes that the test is not 100 per cent reliable.
This is much stronger than "Jacob states that... " since concedes includes the judgement that Jacob was reluctant to make the acknowledgement.
Some other strong reporting words are:
describe, contend, examine, assert, dispute, claim, purport, persuade, refute, concur, recommend, object, dismiss, contradict, propose, examine, observe
\bibliography{refs}

\end{document}